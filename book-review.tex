\newpage\nuluj
\pagestyle{scott}

\autorb{JAMES C. SCOTT}
\nadpisb{Two Cheers for Anarchism:\\
Six Easy Pieces on Autonomy, Dignity, and Meaningful Work and Play}
\vyd{Princeton University Press, 2012}{ISBN 978-0691155296}
\uvod{book-scott-02.png}

Impelled towards anarchism from an unusual direction by his experience studying agrarian communities in Southeast Asia, in \fntit{Two Cheers for Anarchism}, Scott presents an informal perspective on autonomy, dignity, work, play, organisation, and related ideas. He is not driven by theory, or by any large-scale political project, and takes a~pragmatic approach, accepting that states may now be unavoidable. 

Made up of twenty-nine ``fragments" which could mostly stand in isolation, the six ``pieces" in \fntit{Two Cheers for Anarchism} are themselves somewhat anarchic in organisation. 

The first piece begins with blind obedience to red lights by East German pedestrians, considers the importance of small, anonymous acts of insubordination and law-breaking more generally, and works up to the role of massive, spontaneous militancy, outside the co-option of unions or political parties, in key historical and political events. It also contains a~fragment on the importance of crowd feedback in the workings of some kinds of charisma. 

In the second piece, Scott contrasts vernacular and official order. Some of this draws on examples from his own speciality, agriculture, comparing the problems of monoculture plantations with a~Guatemalan peasant garden which is apparently disordered but actually highly efficient. He also contrasts informal names and so forth with international homogenization in language, culture, political forms, and modes of sensibility. 

Piece three, on ``The Production of Human Beings", starts with an adventure playground in Emdrup, Denmark and the Vietnam Memorial in Washington, considers notions of efficiency and alternative approaches to a~Gross Human Product, touches on the authoritarianism of convalescent homes and other pathologies of the institutional life, and ends with the removal of red lights and street signs in Drachten, the Netherlands. 

Piece four is a~defence of that much maligned class, the petty bourgeoisie (in which Scott includes peasants). A~key feature of the class is that members are largely in control of their work; Scott highlights the attraction of the autonomy and self-respect and freedom that brings. He also warns of the co-option of this by businesses that attempt to provide workers with the illusion of autonomy while denying them any real freedom. 

\clearpage\newpage

The fifth piece turns to the misuse of quantification in measurements of scholastic and academic performance, notably in SAT tests and citation indices. These approaches pretend to avoid politics while concealing their agenda in the structure of the system imposed. 

And the short final piece, ``Particularity and Flux", begins with how the town of Le Chambon-sur-Lignon sheltered Jews in the Second World War, then looks more broadly at the rewriting of messy, contingent events as neat history and at how states misrepresent historical events in the service of theatrical bluster and symbolic pageantry. 

Occasionally, traces of academic style and terminology come through in Scott's prose. For example: ``The order, rationality, abstractness, and synoptic legibility of certain kinds of schemes of naming, landscape, architecture, and work processes lend themselves to hierarchical power." But mostly he achieves a~clear, accessible language, and the pieces in \fntit{Two Cheers for Anarchism} should attract a~broader audience than his scholarly books. One hopes they will bring people to think again about some of the assumptions of both ordinary life and political culture. 

